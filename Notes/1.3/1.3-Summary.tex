\documentclass{exam}

\usepackage{geometry}
\usepackage{amssymb}
\usepackage{amsmath}
\usepackage{amsthm}
\usepackage{enumerate}
\usepackage{tcolorbox}
\usepackage{multicol}
\usepackage{parskip}

\usepackage[english]{babel}


\geometry{margin=1in, a4paper}

\newcommand{\rn}{\mathbb{R}}
\newcommand{\abs}[1]{\left|#1\right|}

\newcommand{\answer}{\item}

\newenvironment{answers}[1]
    {\begin{multicols}{#1}
        \begin{enumerate}[a)]
    }
    {
        \end{enumerate}
        \end{multicols}
    }


\title{\Huge{Chapter 1.3 - Summary}\\\huge{Properties of Graphs and Functions}}
\author{Cole Kauder-McMurrich}
\date{\today}

\begin{document}
    \maketitle

    You can characterize functions based on the following properties:
    \begin{itemize}
        \item Domain and Range
        \item Zeros and y-intercepts
        \item Continuity and discontinuity
        \item Intervals of increase and decrease
        \item Symmetry
        \item End behaviour
    \end{itemize}

    Lets now define what those are actually.

    \textbf{Interval of increase}: The intervals(sections) of the domain where the output is increasing, from left to right.

    \textbf{Interval of decrease}: The intervals(sections) of the domain where the output is decreasing, from left to right.

    State the intervals of increase and decrease for the function \(x \mapsto x^2 \)

    Interval of increase: \((0, \infty)\),
    Interval of decrease: \((\infty, 0)\)

    \textbf{Continuous Function}: Any function that has a fully define domain(has no breaks or holes).

    \textbf{Discontinuity}: A break in the domain.

    \textbf{End behaviour}: The behaviour of a function at end(what is the x and y approaching).

    \textbf{Symmetry}: The symmetry of a function, if a function has even symmetry, it's symmetrical over the y axis. If a function has odd symmetry, it's symmetrical rotational around the origin.

    If a function is odd then \(-f(-x)\) \textbf{MUST} equal \(f(x)\).\\ 
    If a function is even then \(f(-x)\) \textbf{MUST} equal \(f(x)\).

    For example, lets find the symmetry for \(f(x)=\frac{1}{x}\) and \(g(x)=x^2\)
    \begin{multicols}{2}
        \[
            \begin{aligned}
            -f(-x) &= -1\left(\frac{1}{(-x)}\right)\\
                   &= \frac{1}{x}\\
            -f(-x) &= f(x)
            \end{aligned}
        \]

        \(\therefore\) The function \(f(x)\) has odd symmetry.

        \columnbreak
        \[
            \begin{aligned}
            f(-x) &= (-x)^2\\
                   &= x^2\\
            f(-x) &= f(x)
            \end{aligned}
            \]
        \(\therefore\) The function \(f(x)\) has even symmetry. 
    \end{multicols}
\end{document}
