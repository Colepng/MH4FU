\documentclass{exam}

\usepackage{pgfplots}
\pgfplotsset{compat=1.18}

\usepackage{geometry}
\usepackage{amssymb}
\usepackage{amsmath}
\usepackage{amsthm}
\usepackage{enumerate}
\usepackage{tcolorbox}
\usepackage{multicol}
\usepackage{parskip}

\usepackage[english]{babel}


\geometry{margin=1in, a4paper}

\newcommand{\rn}{\mathbb{R}}
\newcommand{\abs}[1]{\left|#1\right|}

\newcommand{\answer}{\item}

\newenvironment{answers}[1]
    {\begin{multicols}{#1}
        \begin{enumerate}[a)]
    }
    {
        \end{enumerate}
        \end{multicols}
    }


\title{\Huge{Chapter 1.2 - Summary}\\\huge{Exploring Absolute Values}}
\author{Cole Kauder-McMurrich}
\date{\today}

\begin{document}
    \maketitle

    The absolute value of a x is denoted as \(|x|\), it's the non negative value of x, aka dropping the sign. \(|x| = x\) if \(x \ge 0 \), \(|x| = -x\) if \(x <  0\).

    Some of the properties of the function is: 
    \begin{itemize}
        \item It has even Symmetry
        \item As \(x \rightarrow \infty,  y \rightarrow \infty\)\hfill\newline
        As \(x \rightarrow -\infty,  y \rightarrow \infty\)
    \item \(D: \{ x \in \rn\} \)
    \item \(R: \{ x \in \rn| y \ge 0 \} \)
    \end{itemize}

\begin{questions}
    \question Evaluate 

    \begin{answers}{2}
        \answer \(\abs{-18} = 18\)
        \answer \(-\abs{-36} = -36\)
    \end{answers}

    \question Express \( x < -5 \text{ OR } x > 5\) using absolute value notation. \hfill\newline
    \( x < -5 \text{ OR } x > 5\) can be expressed as \(|x| > 5\).

\end{questions}
\center{\Huge{\textbf{Add graphs to make it feel more complete, otherwise done .}}}

\end{document}
