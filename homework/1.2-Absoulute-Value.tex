\documentclass{exam}

\usepackage{geometry}
\usepackage{amssymb}
\usepackage{amsmath}
\usepackage{amsthm}
\usepackage{enumerate}
\usepackage{tcolorbox}
\usepackage{multicol}
\usepackage{parskip}

\usepackage[english]{babel}


\geometry{margin=1in, a4paper}

\newcommand{\rn}{\mathbb{R}}
\newcommand{\abs}[1]{\left|#1\right|}

\newcommand{\answer}{\item}

\newenvironment{answers}[1]
    {\begin{multicols}{#1}
        \begin{enumerate}[a)]
    }
    {
        \end{enumerate}
        \end{multicols}
    }


\title{\Huge{Chapter 1.2 - Homework}\\\huge{Absolute Value}}
\author{Cole Kauder-McMurrich}
\date{\today}

\begin{document}
    \maketitle
\begin{questions}
    \question Arrange these values in order, from least to greatest:
    $ \abs{-5}, \abs{20}, \abs{-15}, \abs{12}, \abs{-25} $

    The values from least to greatest is: $ \abs{-5}, \abs{12}, \abs{-15}, \abs{20}, \abs{-25} $

    \question Evaluate.

    \begin{answers}{3}
        \answer \( \abs{-22} = 22 \)
        \answer \( -\abs{-35} = -35 \)
        \answer \( \abs{-5 - 13}=18 \)
        \answer \( \abs{4 - 7} + \abs{-10 + 2} = 11\)
        \answer \(\frac{\abs{-8}}{-4} = -2\)
        \answer \(\frac{\abs{-22}}{\abs{-11}} + \frac{-16}{\abs{-4}} = -2\)
    \end{answers}

    \question Express using absolute value notation.

    \begin{answers}{4}
        \answer \(\abs{x} > 3\)
        \answer \(\abs{x} \le 8\)
        \answer \(\abs{x} \geq 1 \)
        \answer \( \abs{x} \ne 5 \)
    \end{answers}

    \question Graph on a number line.

    Done on paper

    \question Rewrite using absolute value notation.

    \begin{answers}{4}
        \answer \( \abs{x} \leq 3 \)
        \answer \( \abs{x} > 2\)
        \answer \( \abs{x} \ge 2\)
        \answer \( \abs{x} < 4 \)
    \end{answers}

    \addtocounter{question}{1}

    \question Graph the following functions.

    Done on paper

    \question Compare the graph in questions 7. How could you use transformations to describe the graph of \(f(x)=\abs{x+3} -4\)

    Done on paper
    
\end{questions}
\end{document}
